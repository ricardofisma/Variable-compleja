\PassOptionsToPackage{unicode=true}{hyperref} % options for packages loaded elsewhere
\PassOptionsToPackage{hyphens}{url}
%
\documentclass[]{book}
\usepackage{lmodern}
\usepackage{amssymb,amsmath}
\usepackage{ifxetex,ifluatex}
\usepackage{fixltx2e} % provides \textsubscript
\ifnum 0\ifxetex 1\fi\ifluatex 1\fi=0 % if pdftex
  \usepackage[T1]{fontenc}
  \usepackage[utf8]{inputenc}
  \usepackage{textcomp} % provides euro and other symbols
\else % if luatex or xelatex
  \usepackage{unicode-math}
  \defaultfontfeatures{Ligatures=TeX,Scale=MatchLowercase}
\fi
% use upquote if available, for straight quotes in verbatim environments
\IfFileExists{upquote.sty}{\usepackage{upquote}}{}
% use microtype if available
\IfFileExists{microtype.sty}{%
\usepackage[]{microtype}
\UseMicrotypeSet[protrusion]{basicmath} % disable protrusion for tt fonts
}{}
\IfFileExists{parskip.sty}{%
\usepackage{parskip}
}{% else
\setlength{\parindent}{0pt}
\setlength{\parskip}{6pt plus 2pt minus 1pt}
}
\usepackage{hyperref}
\hypersetup{
            pdftitle={Variable compleja},
            pdfauthor={MALLQUI BAÑOS Ricardo Michel},
            pdfborder={0 0 0},
            breaklinks=true}
\urlstyle{same}  % don't use monospace font for urls
\usepackage{longtable,booktabs}
% Fix footnotes in tables (requires footnote package)
\IfFileExists{footnote.sty}{\usepackage{footnote}\makesavenoteenv{longtable}}{}
\usepackage{graphicx,grffile}
\makeatletter
\def\maxwidth{\ifdim\Gin@nat@width>\linewidth\linewidth\else\Gin@nat@width\fi}
\def\maxheight{\ifdim\Gin@nat@height>\textheight\textheight\else\Gin@nat@height\fi}
\makeatother
% Scale images if necessary, so that they will not overflow the page
% margins by default, and it is still possible to overwrite the defaults
% using explicit options in \includegraphics[width, height, ...]{}
\setkeys{Gin}{width=\maxwidth,height=\maxheight,keepaspectratio}
\setlength{\emergencystretch}{3em}  % prevent overfull lines
\providecommand{\tightlist}{%
  \setlength{\itemsep}{0pt}\setlength{\parskip}{0pt}}
\setcounter{secnumdepth}{5}
% Redefines (sub)paragraphs to behave more like sections
\ifx\paragraph\undefined\else
\let\oldparagraph\paragraph
\renewcommand{\paragraph}[1]{\oldparagraph{#1}\mbox{}}
\fi
\ifx\subparagraph\undefined\else
\let\oldsubparagraph\subparagraph
\renewcommand{\subparagraph}[1]{\oldsubparagraph{#1}\mbox{}}
\fi

% set default figure placement to htbp
\makeatletter
\def\fps@figure{htbp}
\makeatother

\usepackage{booktabs}
\usepackage[]{natbib}
\bibliographystyle{apalike}

\title{Variable compleja}
\author{MALLQUI BAÑOS Ricardo Michel}
\date{2020-03-06}

\usepackage{amsthm}
\newtheorem{theorem}{Teorema}[chapter]
\newtheorem{lemma}{Lema}[chapter]
\newtheorem{corollary}{Corolario}[chapter]
\newtheorem{proposition}{Proposición}[chapter]
\newtheorem{conjecture}{Conjectura}[chapter]
\theoremstyle{definition}
\newtheorem{definition}{Definición}[chapter]
\theoremstyle{definition}
\newtheorem{example}{Ejemplo}[chapter]
\theoremstyle{definition}
\newtheorem{exercise}{Ejercicio}[chapter]
\theoremstyle{remark}
\newtheorem*{remark}{Observación}
\newtheorem*{solution}{Solución}
\let\BeginKnitrBlock\begin \let\EndKnitrBlock\end
\begin{document}
\maketitle

{
\setcounter{tocdepth}{1}
\tableofcontents
}
\hypertarget{nuxfameros-complejos}{%
\chapter{Números complejos}\label{nuxfameros-complejos}}

Los números complejos son de importancia en el campo de las matemáticas y el desarollo de campos un poco más elavorados

\hypertarget{el-uxe1lgebra-de-los-nuxfameros-complejos}{%
\section{El álgebra de los números complejos}\label{el-uxe1lgebra-de-los-nuxfameros-complejos}}

\BeginKnitrBlock{definition}[Número complejo]
\protect\hypertarget{def:unnamed-chunk-1}{}{\label{def:unnamed-chunk-1} \iffalse (Número complejo) \fi{} }Un número complejo es una expresión de la forma \(a+bi\) donde \(a\) y \(b\) son números reales
\EndKnitrBlock{definition}

\BeginKnitrBlock{theorem}
\protect\hypertarget{thm:unnamed-chunk-2}{}{\label{thm:unnamed-chunk-2} }Sea \(\epsilon\)
\EndKnitrBlock{theorem}

\BeginKnitrBlock{proof}
\iffalse{} {Demostración. } \fi{}En efecto dado \(a>b\) entonces
\EndKnitrBlock{proof}

\hypertarget{el-plano-extendido}{%
\section{El plano extendido}\label{el-plano-extendido}}

\hypertarget{ejercicios}{%
\section{Ejercicios}\label{ejercicios}}

\BeginKnitrBlock{exercise}
\protect\hypertarget{exr:unnamed-chunk-4}{}{\label{exr:unnamed-chunk-4} }
\EndKnitrBlock{exercise}

\BeginKnitrBlock{solution}
\iffalse{} {Solución. } \fi{}
\EndKnitrBlock{solution}

\hypertarget{teoruxeda-de-funciones-mathbbc-diferenciables}{%
\chapter{\texorpdfstring{Teoría de funciones \(\mathbb{C}\)-diferenciables}{Teoría de funciones \textbackslash{}mathbb\{C\}-diferenciables}}\label{teoruxeda-de-funciones-mathbbc-diferenciables}}

\hypertarget{introduccion}{%
\section{Introduccion}\label{introduccion}}

\hypertarget{funciones-c-diferenciables-y-holomorfas}{%
\section{Funciones C-diferenciables y holomorfas}\label{funciones-c-diferenciables-y-holomorfas}}

\hypertarget{series-de-potencias-y-funciones-holomorfas}{%
\section{Series de Potencias y funciones holomorfas}\label{series-de-potencias-y-funciones-holomorfas}}

\hypertarget{ejercicios-1}{%
\section{Ejercicios}\label{ejercicios-1}}

\hypertarget{literature}{%
\chapter{Literature}\label{literature}}

\hypertarget{potencias-y-rauxedces}{%
\section{Potencias y raíces}\label{potencias-y-rauxedces}}

\hypertarget{la-fuxf3rmula-de-euler}{%
\section{La fórmula de Euler}\label{la-fuxf3rmula-de-euler}}

\hypertarget{las-funciones-exponencial-y-logaritmo}{%
\section{Las funciones exponencial y logaritmo}\label{las-funciones-exponencial-y-logaritmo}}

\hypertarget{la-funciuxf3n-exponencial}{%
\subsection{La función exponencial}\label{la-funciuxf3n-exponencial}}

\hypertarget{la-funciuxf3n-logaritmo}{%
\subsection{La función Logaritmo}\label{la-funciuxf3n-logaritmo}}

\hypertarget{las-fuciones-trigonomuxe9tricas}{%
\section{Las fuciones trigonométricas}\label{las-fuciones-trigonomuxe9tricas}}

\hypertarget{aplicaciones-conformes}{%
\chapter{Aplicaciones Conformes}\label{aplicaciones-conformes}}

\hypertarget{introducciuxf3n}{%
\section{Introducción}\label{introducciuxf3n}}

\hypertarget{aplicaciones-conformes-1}{%
\section{Aplicaciones Conformes}\label{aplicaciones-conformes-1}}

\hypertarget{transformaciones-de-mobius}{%
\section{Transformaciones de M¨obius}\label{transformaciones-de-mobius}}

\hypertarget{simetruxeda}{%
\section{Simetría}\label{simetruxeda}}

\hypertarget{ejercicios-2}{%
\section{Ejercicios}\label{ejercicios-2}}

\hypertarget{integral-de-luxednea}{%
\chapter{Integral de Línea}\label{integral-de-luxednea}}

\hypertarget{integracion-compleja}{%
\section{Integracion Compleja}\label{integracion-compleja}}

\hypertarget{el-teorema-de-cauchy}{%
\section{El Teorema de Cauchy}\label{el-teorema-de-cauchy}}

\hypertarget{ejercicios-3}{%
\section{Ejercicios}\label{ejercicios-3}}

\hypertarget{funciones-holomorfas}{%
\chapter{Funciones holomorfas}\label{funciones-holomorfas}}

\hypertarget{homotopuxeda-y-el-teorema-de-cauchy}{%
\chapter{Homotopía y el Teorema de Cauchy}\label{homotopuxeda-y-el-teorema-de-cauchy}}

\hypertarget{el-indice-de-una-curva}{%
\chapter{El Indice de una curva}\label{el-indice-de-una-curva}}

\bibliography{book.bib,packages.bib}

\end{document}
